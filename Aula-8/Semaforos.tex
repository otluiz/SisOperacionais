\documentclass[11pt]{beamer}
\usetheme{Warsaw}
\usepackage[utf8]{inputenc}
\usepackage[brazil]{babel}  % idioma
\usepackage{amsmath,amsfonts,amssymb,textcomp}
\usepackage{graphicx}
\usepackage{subfigure}

\usepackage[lined]{algorithm2e}

% Configurando layout para mostrar codigos C++
\usepackage{listings}
%\usepackage{minted}
%pdflatex -shell-escape minted01.tex
%ou
%latexmk -pdf -shell-escape minted01.tex

\author{Othon Oliveira}
\title{Sistemas Operacionais}
%\setbeamercovered{transparent} 
%\setbeamertemplate{navigation symbols}{} 
%\logo{} 
\institute{Fatec -- Faculdade de Informática --- PE} 
%\date{} 
%\subject{} 
\begin{document}


% Capa - requer o TikZ
\newcommand{\capa}{
    \begin{tikzpicture}[remember picture,overlay]
        \node at (current page.south west)
            {\begin{tikzpicture}[remember picture, overlay]
                \fill[shading=radial,top color=orange,bottom color=orange,middle color=yellow] (0,0) rectangle (\paperwidth,\paperheight);
            \end{tikzpicture}
          };
    \end{tikzpicture}
}




\begin{frame}
\titlepage
\end{frame}

\begin{frame}
\tableofcontents
\end{frame}



%+++++++++++++++++++++++++++++++++++++++++++++++
\section{ Semáforos}
\subsection*{ E. W. Dijkstra (1965)}

\begin{frame}\frametitle{ Contando os sinais de ``acordar''}
 Dijkstra sugeriu uma variável inteira para contar o número de sinais de acordar salvos, para uso futuro.
 
\pause
\begin{block}{ Semáforos}
  Essa proposta introduziu uma nova variável chamada de \textbf{semáforos}. Um semáforo poderia conter o valor 0, indicando que nenhum sinal 
  de acordar foi salvo -- ou algum valor positivo se um sinal de acordar estivessem pendentes.
\end{block}
 
 \pause
\begin{block}{ Up e Down}
 Dijkstra propôs a existência de duas operações, \textit{down} e \textit{up} (generalização de sleep e wakeup respectivamente). 
\end{block}
 
\end{frame}


%--------------------------------------------------------
\begin{frame}\frametitle{ Ações sobre semáforos}

\begin{block}{ Operações de semáforos}
  A operação \textit{down} sobre um semáforo verifica se seu valor é maior que 0, se for decresce de 1 
  (equivale a dizer que ``gasta'' um sinal de acordar armazenado) e prosseguirá, caso seja 0 o processo será posto pra dormir.
\end{block}

\pause
\begin{block}{ Ações atômicas}
  Todas essas tarefas são atômicas, ou seja indivisíveis. Garante que, uma vez iniciada uma operação de semáforo, nenhum 
  outro processo pode ter acesso ao semáforo até que a operação seja terminada ou bloqueada.
\end{block}

\pause
  Essas ações atômicas são absolutamente essenciais para resolver o problema da sincronização e evitar condições de disputa.

\end{frame}


%--------------------------------------------------------
\begin{frame}\frametitle{ Ações sobre semáforos}

\begin{block}{ Ações \textit{up} e \textit{down}}
  A operação \textit{up} incrementa o valor de um semáforo. Se um ou mais processos estiverem dormindo naquele semáforo, 
  incapacitados de terminar um operação \textit{down} anterior, um deles seria escolhido pelo sistema (ex: aleatoriamente) 
  e seria dada permissão para executar seu \textit{down}.
\end{block}

\pause
\begin{block}{ Indivisibilidade}
  A operação de incrementar o semáforo e acordar um processo é também indivisível. 
  Nunca um processo é bloqueado a partir de um \textit{up} (princípio 3 das condições de disputa)
\end{block}
\end{frame}


%+++++++++++++++++++++++++++++++++++++++++++++++
\subsection*{ O algoritmo produtor--concumidor com semáforos}
\begin{frame}\frametitle{ O problema da perda do sinal de ``acordar''}

\begin{block}{ Indivisibilidade}
 É fundamental que seja inplementada de maneira indivisível, diferentemente do modo normal baseado em \textit{up} e \textit{down}, onde 
 o sistema operacional desabilita suas interrupções, pondo-o para dormir, enquanto estiver testando o semáforo.
\end{block}

\pause
\begin{block}{ O algoritmo em C}
   \begin{algorithm}[H]
    \#define N 100		/* número lugares no buffer */ \\
    typedef int semaphore;	/* semáforos são um tipo especial de int */ \\
    semaphore mutex = 1;	/* controla o acesso à região crítica */\\
    semaphore empty = N;	/* conta os lugares vazios no buffer */\\
    semaphore full = 0;		/* conta os lugares preenchidos no buffer */\\
   ...
  \end{algorithm}
\end{block}

\end{frame}


%------------------------------------------------
\begin{frame}{ Implementando o código do produtor}

 \begin{algorithm}[H]    
    void producer(void)\{ \\
\hspace{0,2 cm}      int item;\\
\vspace{0,3 cm}      
\hspace{0,2 cm}      while(true)\{		/* True é a constante 1 */\\
\hspace{0,5 cm}	item = produzirItem(); 	/* gera o próximo item */\\ 
\hspace{0,5 cm}	down(\&empty);		/* decresce o contador de empty */\\
\hspace{0,5 cm}	down(\&mutex);		/* entra na região crítica */\\
\hspace{0,5 cm}	inserirItem(item);	/* ponha um item no buffer */\\
\hspace{0,5 cm}	up(\&mutex);	 	/* o buffer está vazio? */\\
\hspace{0,5 cm}	up(\&full);		/* incrementa o contador de lugares preenchidos */\\
\hspace{0,2 cm}      \}\\
    \}\\
 \end{algorithm}
\end{frame}


%------------------------------------------------------------

\begin{frame}{ O consumidor}
  \begin{algorithm}[H]
    
    void consumer(void)\{ \\
\hspace{0,2 cm}      int intem; \\
\vspace{0,3 cm}      
\hspace{0,2 cm}      while(true)\{		/* True é a constante 1 */\\
\hspace{0,5 cm}	down(\&full);		/* decresce o contador de full */\\
\hspace{0,5 cm}	down(\&mutex);		/* entra na região crítica */\\
\hspace{0,5 cm}	item = removeItem(); 	/* pega um item do buffer */\\
\hspace{0,5 cm}	up(\&mutex);	 	/* deixa a região crítica */\\
\hspace{0,5 cm}	up(\&empty);		/* incrementa o cont. de lugares preenchidos */\\
\hspace{0,5 cm}	consomeItem(Item);	/* faz algo com um item, ex: imprime */
\hspace{0,2 cm}      \}\\
    \}\\
  \end{algorithm}

\end{frame}

%------------------------------------------------------------------


%------------------------------------------------------------------


\end{document}